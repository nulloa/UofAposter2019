\documentclass[a4paper]{article}
%\usepackage{simplemargins}

%\usepackage[square]{natbib}
\usepackage{amsmath}
\usepackage{amsfonts}
\usepackage{amssymb}
\usepackage{graphicx}

\begin{document}
\pagenumbering{gobble}

\Large
 \begin{center}
Bayesian Functional Data Analysis and Influenza\\ 

\hspace{10pt}

% Author names and affiliations
\large
Nehemias Ulloa, Jarad Niemi \\%$^1$ \\

\hspace{10pt}

\small  
Iowa State University\\
nulloa1@iastate.edu\\

\end{center}

\hspace{10pt}

\normalsize

Influenza is a common illness which affects most people at some point in their lives. At best its a minor inconvenience, but as seen in the past winter season, influenza's intensity is difficult to predict and can cause serious health problems especially among the young, elderly and pregnant women. This past season has highlighted the importance of being able to understand and predict the influenza season. In this poster, a Bayesian hierarchical structure is used in a functional data framework to model the influenza season. Sparse priors are used and compared to choose the number of basis functions. Multiple hierarchical structures are compared to learn more about the structure of influenza season.

\end{document}

